\documentclass[paper=a4, fontsize=11pt]{scrartcl}

\usepackage[utf8]{inputenc}
\usepackage{fourier}
\usepackage[cache=false]{minted}


\usepackage[francais]{babel} %% installer texlive-lang-french pour cela


\title{Projet 2 : Fouine}
\author{Guillaume Duboc, Pierre Oechsel}
\date{\today}


\begin{document}

\maketitle


\section{Présentation}

Nous avons programmé en Ocaml, avec des scripts de test en bash.
Pour le parseur, on utilise lex et yacc via leurs libs Ocaml.

Nous exposons à la partie~\ref{s:orga} comment notre programme est structuré.


Un peu de maths en \LaTeX: voici un exemple de formule~:
$$
\sum_{i\geq 0} \litt_1\lor\non{\litt_2}\lor\litt_4
$$
On remarque au passage que $\non{\non{\litt}}$ est pareil que $\litt$.

\section{Organisation du code}
\label{s:orga}

Le code est structuré de la manière suivante~:
\begin{itemize}
\item bli
\item bla
\item blo
\item Digression à propos des Mustélidés.
\end{itemize}

\section{Critique des performances}

On constate que blibla.


On est par ailleurs capable de citer des références, ainsi~: \cite{ProjInt16}.



On dispose d'un environnement de base contenant nos opérateurs binaires et quelques fonctions utilitaires : .
\begin{minted}{ocaml}
>>> ref;;
- : '_a -> '_a ref = <fun>
>>> (:=);;
- : '_a ref -> '_a -> unit = <fun>
\end{minted}

\begin{minted}{ocaml}
>>> type ('a, 'b) pair = Left of 'a | Right of 'b;;
>>> type 'a tree = Leaf | Node of 'a * 'a tree * 'a tree;;
>>> Leaf;;
- : '_a tree = Leaf
\end{minted}
\begin{minted}{ocaml}
>>> type 'a list = [] | (::) of 'a * 'a list;;
>>> [1; 2; 3]
- : int list = (::) (1, (::) (2, (::) (3, [])))
\end{minted}

\end{document}
% ci-dessous: commenté car non offert sur les machines libre-service.
% décommentez si vous le souhaitez.
%\usepackage[french]{babel}

% pour compiler: 

% faire    pdflatex ex-rapport
% (si les references aux numeros de parties apparaissent comme des
% "?", recompiler une fois)

% la compilation de la bibliographie est davantage une "incantation":
% faire     bibtex ex-biblio
% puis      pdflatex ex-rapport (un nombre premier de fois)


% vous pouvez ensuite ouvrir le fichier ex-rapport.pdf




